Here is some information on the correct syntax to use in order to comment your code to be recognised by the Doxygen system.

See \href{http://www.stack.nl/~dimitri/doxygen/manual/docblocks.html}{\tt here} for full details


\begin{DoxyItemize}
\item \mbox{\hyperlink{c_syntax}{Syntax for C style Languages}}
\item \mbox{\hyperlink{python_syntax}{Syntax for Python}}
\item \mbox{\hyperlink{vhdl_syntax}{Syntax for V\+H\+DL}} 
\end{DoxyItemize}\hypertarget{cSyntax}{}\section{c\+Syntax}\label{cSyntax}
The following Syntax is used to document C/\+C++/C\#/\+Objective-\/\+C/\+P\+H\+P/\+Java languages

You can use the Java\+Doc style, which consist of a C-\/style comment block starting with two $\ast$\textquotesingle{}s, like this\+: 
\begin{DoxyCode}
/**
 * ... text ...
 */
\end{DoxyCode}
 or you can use the Qt style and add an exclamation mark (!) after the opening of a C-\/style comment block, as shown in this example\+: 
\begin{DoxyCode}
/*!
 * ... text ...
 */
\end{DoxyCode}
 In both cases the intermediate $\ast$\textquotesingle{}s are optional, so 
\begin{DoxyCode}
/*!
 ... text ...
*/
\end{DoxyCode}
 is also valid.

A third alternative is to use a block of at least two C++ comment lines, where each line starts with an additional slash or an exclamation mark. Here are examples of the two cases\+: 
\begin{DoxyCode}
///
/// ... text ...
///
\end{DoxyCode}
 or 
\begin{DoxyCode}
//!
//!... text ...
//!
\end{DoxyCode}
 Note that a blank line ends a documentation block in this case.

Some people like to make their comment blocks more visible in the documentation. For this purpose you can use the following\+: 
\begin{DoxyCode}
/********************************************//**
 *  ... text
 ***********************************************/
\end{DoxyCode}
 (note the 2 slashes to end the normal comment block and start a special comment block).

or 
\begin{DoxyCode}
/////////////////////////////////////////////////
/// ... text ...
/////////////////////////////////////////////////
\end{DoxyCode}
 \hypertarget{pythonSyntax}{}\section{python\+Syntax}\label{pythonSyntax}
The following Syntax is used to document Python

\`{}\`{}\`{}python \subsection*{}\hypertarget{vhdlSyntax}{}\section{vhdl\+Syntax}\label{vhdlSyntax}
